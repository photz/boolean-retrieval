\documentclass{beamer}

\usetheme{Frankfurt}

\usepackage{ifluatex}
\ifluatex
  \usepackage{pdftexcmds}
  \makeatletter
  \let\pdfstrcmp\pdf@strcmp
  \let\pdffilemoddate\pdf@filemoddate
  \makeatother
\fi

\usepackage{color, colortbl}
\usepackage{svg}
\usepackage{listings}
\usepackage{mathtools}
%\usepackage{amsmath}

\definecolor{LightCyan}{rgb}{0.88,1,1}
\definecolor{Yellow}{rgb}{1, 1, 0.45}
\definecolor{Green}{rgb}{0.3, 0.9, 0.35}

\mode<presentation>

\title{Boolean Retrieval}

\author{Ehab Abou Asali\\ Philipp Hotz}

\begin{document}


\begin{frame}

  \frametitle{Outline}
  \tableofcontents[pausesections]
\end{frame}


\section{The Program We Wrote}

\subsection{Phrase Search}

\begin{frame}
  \frametitle{Phrase Search}

  \begin{block}{Document Collection}

    \begin{tabular}{ l | l }
      DocID & Contents \\ \hline
      1 & The python ate the apple \\
      2 & Did the python eat the apple? \\
    \end{tabular}

  \end{block}

\end{frame}

\begin{frame}
  \frametitle{Phrase Search}

  \begin{block}{Document Collection}

    \begin{tabular}{ l | l }
      DocID & Contents \\ \hline
      1 & The python ate the apple \\
      2 & Did the python eat the apple? \\
    \end{tabular}

  \end{block}

  \begin{block}{Postings Lists}
    \begin{tabular}{ l | l }
      term & postings \\ \hline
      the & $ \langle 1,(1,4) \rangle \langle 2,(2,5) \rangle $ \\
      python & $ \langle 1,(2) \rangle \langle 2,(3) \rangle $ \\
      ate & $ \langle 1,(3) \rangle $ \\
      apple & $ \langle 1,(5) \rangle \langle 2,(6) \rangle $ \\
      eat & $ \langle 1,(4) \rangle $ \\
      did & $ \langle 2,(1) \rangle $ \\
    \end{tabular}
  \end{block}

\end{frame}

\begin{frame}
  \frametitle{Phrase Search}

    \begin{block}{Document Collection}

    \begin{tabular}{ l | l }
      DocID & Contents \\ \hline
      1 & The python ate the apple \\
      2 & Did the python eat the apple? \\
    \end{tabular}

  \end{block}

  \begin{block}{Postings Lists}
    \begin{tabular}{ l | l }
      term & postings \\ \hline
      the & $ \langle 1,(1,4) \rangle \langle 2,(2,5) \rangle $ \\
      python & $ \langle 1,(2) \rangle \langle 2,(3) \rangle $ \\
      ate & $ \langle 1,(3) \rangle $ \\
      apple & $ \langle 1,(5) \rangle \langle 2,(6) \rangle $ \\
      eat & $ \langle 1,(4) \rangle $ \\
      did & $ \langle 2,(1) \rangle $ \\
    \end{tabular}
  \end{block}
\end{frame}

\begin{frame}
  \frametitle{Phrase Search}

  \begin{block}{Query}
    ``the apple''
  \end{block}

\end{frame}

\begin{frame}
  \frametitle{Phrase Search}

  \begin{block}{Query}
    ``the apple''
  \end{block}

  \begin{block}{Postings Lists}
    \begin{tabular}{ l | l }
      term & postings \\ \hline
      the & $ \langle 1,(1,\colorbox{Yellow}{4}) \rangle \langle 2,(2,\colorbox{Green}{5}) \rangle $ \\
      %python & $ \langle 1,(2) \rangle \langle 2,(3) \rangle $ \\
      %ate & $ \langle 1,(3) \rangle $ \\
      apple & $ \langle 1,(\colorbox{Yellow}{5}) \rangle \langle 2,(\colorbox{Green}{6}) \rangle $ \\
      %eat & $ \langle 1,(4) \rangle $ \\
      %did & $ \langle 2,(1) \rangle $ \\
    \end{tabular}
  \end{block}
\end{frame}

\begin{frame}
  \frametitle{Phrase Search}

  \begin{block}{Query}
    ``the apple''
  \end{block}

  \begin{block}{Relevant Postings Lists}
    \begin{tabular}{ l | l }
      term & postings \\ \hline
      the & $ \langle 1,(1,\colorbox{Yellow}{4}) \rangle \langle 2,(2,\colorbox{Green}{5}) \rangle $ \\
      %python & $ \langle 1,(2) \rangle \langle 2,(3) \rangle $ \\
      %ate & $ \langle 1,(3) \rangle $ \\
      apple & $ \langle 1,(\colorbox{Yellow}{5}) \rangle \langle 2,(\colorbox{Green}{6}) \rangle $ \\
      %eat & $ \langle 1,(4) \rangle $ \\
      %did & $ \langle 2,(1) \rangle $ \\
    \end{tabular}
  \end{block}

  \begin{block}{Results}

    \begin{tabular}{ l | l }
      DocID & Contents \\ \hline
      1 & The python ate \colorbox{Yellow}{the apple} \\
      2 & Did the python eat \colorbox{Green}{the apple}? \\
    \end{tabular}

  \end{block}
\end{frame}


\defverbatim[colored]\lstPostingsList{
  \begin{lstlisting}[language=Java,basicstyle=\ttfamily,keywordstyle=\color{red}]
    PostingsList the, tasty, apple;

    the.posIntersect(tasty)
       .posIntersect(apple);
  \end{lstlisting}
}

\begin{frame}
  \frametitle{Using PostingsList objects}

  \begin{example}
    \lstPostingsList
  \end{example}
\end{frame}


\subsection{UML}

\begin{frame}
  \frametitle{UML Class Diagram}
  \begin{figure}[htbp]
    \centering
    \includesvg{uml}
  \end{figure}
\end{frame}

% two kinds of XML parsers
\subsection{Reading the XML file}

\begin{frame}
  \frametitle{Two Kinds of XML Parsers}

  \begin{columns}
    \column{0.5\textwidth}
    \begin{block}{Streaming}
      \begin{itemize}
        \item the file is read from beginning to end in a single pass
        \item low memory footprint
      \end{itemize}
    \end{block}
    \column{0.5\textwidth}
    \begin{block}{DOM}
      \begin{itemize}
      \item the entire tree is built in memory
      \item memory footprint potentially bigger than the document itself
      \end{itemize}
    \end{block}
  \end{columns}
\end{frame}


\begin{frame}[fragile]

  \frametitle{XML}

\begin{verbatim}
<quiz>
 <qanda>
  <question>
    Who was the third president of the U.S.A.?
  </question>
  <answer>Err .. Homer Simpson?</answer>
 </qanda>
</quiz>
\end{verbatim}

\end{frame}

\defverbatim[colored]\lstReadXml{
  \begin{lstlisting}[language=Java]
    while (EOF_not_yet_reached) {
      switch (type_of_event) {
        case XMLStreamConstants.START_ELEMENT:
        if ("MedlineCitation.equals(tagname)) {
          // ...
        }
        break;

        case XMLStreamConstants.END_ELEMENT:
        break;

        case XMLStreamConstants.CHARACTER:

        breal;
      }
    }
  \end{lstlisting}
}

\begin{frame}[fragile]

  \frametitle{Code Example}

  \lstReadXml

\end{frame}


% what's the coverage?
\subsection{Unit Tests}

\begin{frame}
  \frametitle{Unit Tests}

  \begin{block}{Test Coverage}
    \begin{tabular}{ l | l }
      Class & Number of test cases \\ \hline
      Index & 22 \\
      PostingsList & 11 \\
      DocumentCollection & 8 \\
    \end{tabular}
  \end{block}
\end{frame}

\subsection{Caching the Index}

\begin{frame}
  \frametitle{Caching the Index}

  \begin{block}{Postings Lists}
    \begin{tabular}{ l | l }
      term & postings \\ \hline
      the & $ \langle 1,(1,4) \rangle \langle 2,(2,5) \rangle $ \\
      python & $ \langle 1,(2) \rangle \langle 2,(3) \rangle $ \\
      ate & $ \langle 1,(3) \rangle $ \\
      apple & $ \langle 1,(5) \rangle \langle 2,(6) \rangle $ \\
      eat & $ \langle 1,(4) \rangle $ \\
      did & $ \langle 2,(1) \rangle $ \\
    \end{tabular}
  \end{block}

\end{frame}


\begin{frame}
  \frametitle{Caching the Index}

  \begin{columns}

    \column{0.5\textwidth}

    \begin{block}{Postings Lists}
      \begin{tabular}{ l | l }
        term & postings \\ \hline
        \rowcolor{LightCyan}
        the & $ \langle 1,(1,4) \rangle \langle 2,(2,5) \rangle $ \\
        python & $ \langle 1,(2) \rangle \langle 2,(3) \rangle $ \\
        ate & $ \langle 1,(3) \rangle $ \\
        apple & $ \langle 1,(5) \rangle \langle 2,(6) \rangle $ \\
        eat & $ \langle 1,(4) \rangle $ \\
        did & $ \langle 2,(1) \rangle $ \\
      \end{tabular}
    \end{block}

    \column{0.5\textwidth}

    \begin{block}{Positions Table}
      \begin{tabular}{ l | l }
        term & position in binary file \\ \hline
        \rowcolor{LightCyan}
        the & 0 \\
      \end{tabular}
    \end{block}
    
  \end{columns}

  \begin{block}{Cache}
    \colorbox{LightCyan}{$ \langle 1,(1,4) \rangle \langle 2,(2,5) \rangle $}
  \end{block}
\end{frame}

\begin{frame}
  \frametitle{Caching the Index}

  \begin{columns}

    \column{0.5\textwidth}

    \begin{block}{Postings Lists}
      \begin{tabular}{ l | l }
        term & postings \\ \hline
        the & $ \langle 1,(1,4) \rangle \langle 2,(2,5) \rangle $ \\
        \rowcolor{LightCyan}
        python & $ \langle 1,(2) \rangle \langle 2,(3) \rangle $ \\
        ate & $ \langle 1,(3) \rangle $ \\
        apple & $ \langle 1,(5) \rangle \langle 2,(6) \rangle $ \\
        eat & $ \langle 1,(4) \rangle $ \\
        did & $ \langle 2,(1) \rangle $ \\
      \end{tabular}
    \end{block}

    \column{0.5\textwidth}

    \begin{block}{Positions Table}
      \begin{tabular}{ l | l }
        term & position in binary file \\ \hline
        the & 0 \\
        \rowcolor{LightCyan}
        python & 40 \\
      \end{tabular}
    \end{block}
    
  \end{columns}

  \begin{block}{Cache}
    $ \langle 1,(1,4) \rangle \langle 2,(2,5) \rangle $
    \colorbox{LightCyan}{$ \langle 1,(2) \rangle \langle 2,(3) \rangle $}
  \end{block}
\end{frame}

\begin{frame}
  \frametitle{Caching the Index}

  \begin{columns}

    \column{0.5\textwidth}

    \begin{block}{Postings Lists}
      \begin{tabular}{ l | l }
        term & postings \\ \hline
        the & $ \langle 1,(1,4) \rangle \langle 2,(2,5) \rangle $ \\
        python & $ \langle 1,(2) \rangle \langle 2,(3) \rangle $ \\
        \rowcolor{LightCyan}
        ate & $ \langle 1,(3) \rangle $ \\
        apple & $ \langle 1,(5) \rangle \langle 2,(6) \rangle $ \\
        eat & $ \langle 1,(4) \rangle $ \\
        did & $ \langle 2,(1) \rangle $ \\
      \end{tabular}
    \end{block}

    \column{0.5\textwidth}

    \begin{block}{Positions Table}
      \begin{tabular}{ l | l }
        term & position in binary file \\ \hline
        the & 0 \\
        python & 40 \\
        \rowcolor{LightCyan}        
        ate & 90 \\
      \end{tabular}
    \end{block}
    
  \end{columns}

  \begin{block}{Cache}
    $ \langle 1,(1,4) \rangle \langle 2,(2,5) \rangle $
    $ \langle 1,(2) \rangle \langle 2,(3) \rangle $
    \colorbox{LightCyan}{$ \langle 1,(3) \rangle $}
  \end{block}
\end{frame}

\begin{frame}
  \frametitle{Caching the Index}

  \begin{columns}

    \column{0.5\textwidth}

    \begin{block}{Postings Lists}
      \begin{tabular}{ l | l }
        term & postings \\ \hline
        the & $ \langle 1,(1,4) \rangle \langle 2,(2,5) \rangle $ \\
        python & $ \langle 1,(2) \rangle \langle 2,(3) \rangle $ \\
        ate & $ \langle 1,(3) \rangle $ \\
        \rowcolor{LightCyan}
        apple & $ \langle 1,(5) \rangle \langle 2,(6) \rangle $ \\
        eat & $ \langle 1,(4) \rangle $ \\
        did & $ \langle 2,(1) \rangle $ \\
      \end{tabular}
    \end{block}

    \column{0.5\textwidth}

    \begin{block}{Positions Table}
      \begin{tabular}{ l | l }
        term & position in binary file \\ \hline
        the & 0 \\
        python & 40 \\
        ate & 90 \\
        \rowcolor{LightCyan}
        apple & 120 \\
      \end{tabular}
    \end{block}
    
  \end{columns}

  \begin{block}{Cache}
    $ \langle 1,(1,4) \rangle \langle 2,(2,5) \rangle $
    $ \langle 1,(2) \rangle \langle 2,(3) \rangle $
    $ \langle 1,(3) \rangle $
    \colorbox{LightCyan}{$ \langle 1,(5) \rangle \langle 2,(6) \rangle $}
  \end{block}
\end{frame}

\begin{frame}
  \frametitle{Caching the Index}

  \begin{columns}

    \column{0.5\textwidth}

    \begin{block}{Postings Lists}
      \begin{tabular}{ l | l }
        term & postings \\ \hline
        the & $ \langle 1,(1,4) \rangle \langle 2,(2,5) \rangle $ \\
        python & $ \langle 1,(2) \rangle \langle 2,(3) \rangle $ \\
        ate & $ \langle 1,(3) \rangle $ \\
        apple & $ \langle 1,(5) \rangle \langle 2,(6) \rangle $ \\
        \rowcolor{LightCyan}
        eat & $ \langle 1,(4) \rangle $ \\
        did & $ \langle 2,(1) \rangle $ \\
      \end{tabular}
    \end{block}

    \column{0.5\textwidth}

    \begin{block}{Positions Table}
      \begin{tabular}{ l | l }
        term & position in binary file \\ \hline
        the & 0 \\
        python & 40 \\
        ate & 90 \\
        apple & 120 \\
        \rowcolor{LightCyan}
        eat & 170 \\
      \end{tabular}
    \end{block}
    
  \end{columns}

  \begin{block}{Cache}
    $ \langle 1,(1,4) \rangle \langle 2,(2,5) \rangle $
    $ \langle 1,(2) \rangle \langle 2,(3) \rangle $
    $ \langle 1,(3) \rangle $
    $ \langle 1,(5) \rangle \langle 2,(6) \rangle $
    \colorbox{LightCyan}{$ \langle 1,(4) \rangle $}
  \end{block}
\end{frame}

\begin{frame}
  \frametitle{Caching the Index}

  \begin{columns}

    \column{0.5\textwidth}

    \begin{block}{Postings Lists}
      \begin{tabular}{ l | l }
        term & postings \\ \hline
        the & $ \langle 1,(1,4) \rangle \langle 2,(2,5) \rangle $ \\
        python & $ \langle 1,(2) \rangle \langle 2,(3) \rangle $ \\
        ate & $ \langle 1,(3) \rangle $ \\
        apple & $ \langle 1,(5) \rangle \langle 2,(6) \rangle $ \\
        eat & $ \langle 1,(4) \rangle $ \\
        \rowcolor{LightCyan}
        did & $ \langle 2,(1) \rangle $ \\
      \end{tabular}
    \end{block}

    \column{0.5\textwidth}

    \begin{block}{Positions Table}
      \begin{tabular}{ l | l }
        term & position in binary file \\ \hline
        the & 0 \\
        python & 40 \\
        ate & 90 \\
        apple & 120 \\
        eat & 170 \\
        \rowcolor{LightCyan}
        did & 210 \\
      \end{tabular}
    \end{block}
    
  \end{columns}

  \begin{block}{Cache}
    $ \langle 1,(1,4) \rangle \langle 2,(2,5) \rangle $
    $ \langle 1,(2) \rangle \langle 2,(3) \rangle $
    $ \langle 1,(3) \rangle $
    $ \langle 1,(5) \rangle \langle 2,(6) \rangle $
    $ \langle 1,(4) \rangle $
    \colorbox{LightCyan}{$ \langle 2,(1) \rangle $}
  \end{block}
\end{frame}

\section{Difficulties}


%% \defverbatim[colored]\lstI{
%% \begin{lstlisting}[language=C++,basicstyle=\ttfamily,keywordstyle=\color{red}]
%% int main() {
%%   // Define variables at the beginning
%%   // of the block, as in C:
%%   CStash intStash, stringStash;
%%   int i;
%%   char* cp;
%%   ifstream in;
%%   string line;
%% [...]
%% \end{lstlisting}
%% } 

% http://stackoverflow.com/questions/1995113/strangest-language-feature/2001861#2001861
\defverbatim[colored]\lstQuiz{
  \begin{lstlisting}[language=Java,basicstyle=\ttfamily,keywordstyle=\color{red}]
    Integer foo = 1000;
    Integer bar = 1000;

    foo <= bar;
    foo >= bar;
    foo == bar;
  \end{lstlisting}
}



\begin{frame}
  \frametitle{Quiz}
  \begin{block}
    \lstQuiz
    \end{block}
  \begin{block}
    block
  \end{block}
\end{frame}

\section{Feedback}

%% \begin{thebibliography}{10}
%% \bibitem{Goldbach1742}[Goldbach, 1742]
%% Christian Goldbach.
%% \newblock A problem we should try to solve before the ISPN ’43 deadline,
%% \newblock \emph{Letter to Leonhard Euler}, 1742.
%% \end{thebibliography}

\end{document}
